\chapter{Chapter 4\newline Extending the hierarchical Pitman-Yor language models with skipgrams}\label{chap:shpyplm}\marginnote{This section is based on: \cite{onrust2016Improving}}

%\section{Introduction}

\newthought{Since the seminal paper} on hierarchical Bayesian language models based on Pitman-Yor processes\cite{teh2006hierarchical}, Bayesian language modelling has regained an interest. Although Bayesian language models are not new\cite{mackay1995hierarchical}, previously proposed models were reported to be inferior compared to other smoothing methods. Teh's work was the first to report on improvements over interpolated Kneser-Ney smoothing\cite{teh2006hierarchical}.
    
  To overcome the traditional problems of overestimating the probabilities of rare occurrences and underestimating the probabilities of unseen events, a range of smoothing algorithms have been proposed in the literature \cite{goodman2001bit}. Most methods take a heuristic-frequentist approach combining $n$-gram probabilities for various values of $n$, using back-off schemes or interpolation. 

\textcite{teh2006hierarchical} showed that MacKay and Peto's \citep{mackay1995hierarchical} research on parametric Bayesian language models with a Dirichlet prior could be extended to give better results, but also that one of the best smoothing methods, interpolated Kneser-Ney \cite{kneser1995improved}, can be derived as an approximation of the Hierarchical Pitman-Yor process language model (HPYLM).
  
  % Eventueel eruit
  The success of the Bayesian approach to language modelling is due to the use of statistical distributions such as the Dirichlet distribution, and distributions over distributions, such as the Dirichlet process and its two-parameter generalisation, the Pitman-Yor process. Both are widely studied in the statistics and probability theory communities. Interestingly, language modelling has acquired the status of a ``fruit fly'' problem in these communities, to benchmark the performance of statistical models. In this paper we approach language modelling from a computational linguistics point of view, and consider the statistical methods to be the tool with the future goal of improving language models for extrinsic tasks such as speech recognition.
  
  We derive our model from \textcite{teh2006hierarchical}, and propose an extension with skipgrams. A frequentist approach to language modelling with skipgrams is described by \textcite{pickhardt2014generalized}, who introduce an approach using skip-$n$-grams which are interpolated using modified Kneser-Ney smoothing. In this paper we show that a Bayesian skip-$n$-gram approach outperforms a frequentist skip-$n$-gram model.%, 

\section{Method}\marginnote{This section is a recap of \cref{chap:intoblm}, and in particular the Bayesian $n$-gram language model.}

Pitman-Yor Processes (PYP) belong to the family of non-parametric Bayesian models. Let $W$ be a fixed and finite vocabulary of $V$ words. For each word $w\in W$ let $G(w)$ be the probability of $w$, and $G = [G(w)]_{w\in W}$ be the vector of word probabilities. Since word frequencies generally follow a power-law distribution, we use a Pitman-Yor process, which is a distribution over partitions with power-law distributions. 
In the context of a language model this means that for a space $P(\mathbf{u})$, with $c(\mathbf{u}\cdot)$ elements (tokens), we want to partition $P(\mathbf{u})$ in $V$ subsets such that the partition is a good approximation of the underlying data, in which $c(\mathbf{u}w)$ is the size of subset $w$ of $P(\mathbf{u})$. We assume that the training data is an sample of the underlying data, and for this reason we seek to find an approximation, rather than using the partitions precisely as found in the training data.

Since we also assume that a power-law distribution on the words in the underlying data, we place a PYP prior on G:
  \begin{equation*}
  	G \sim \operatorname{PY}(d,\theta,G_0),
  \end{equation*}
with discount parameter $0\leq d<1$, a strength parameter $\theta > -d$ and a mean vector $G_0 = [G_0(w)]_{w\in W}$. $G_0(w)$ is the a-priori probability of word $w$, which we set uniformly: $G_0(w) = 1/V$ for all $w\in W$. In general, there is no known analytic form for the density of $\operatorname{PY}(d,\theta,G_0)$ when the vocabulary is finite. However, we are interested in the distribution over word sequences induced by the PYP, which has a tractable form, and is sufficient for the purpose of language modelling. 

  Let $G$ and $G_0$ be distributions over $W$, and $x_1,x_2,\ldots$ be a sequence of words drawn i.i.d.\ from $G$. The PYP is then described in terms of a generative procedure that takes  $x_1,x_2,\ldots$ to produce a separate sequence of i.i.d.\ draws $y_1,y_2,\ldots$ from the mean distribution $G_0$ as follows. The first word $x_1$ is assigned the value of the first draw $y_1$ from $G_0$. Let $t$ be the current number of draws from $G_0$, $c_k$ the number of words assigned the value of draw $y_k$ and $c_\cdot = \sum^t_{k=1}c_k$ the number of draws from $G_0$. For each subsequent word $x_{c_\cdot+1}$, we either assign it the value of a previous draw $y_k$, with probability $\frac{c_k-d}{\theta+c_\cdot}$, or assign it the value of a new draw from $G_0$ with probability $\frac{\theta+dt}{\theta+c_\cdot}$.

For an $n$-gram language model we use a hierarchical extension of the PYP. The hierarchical extension describes the probabilities over the current word given various contexts consisting of up to $n-1$ words. Given a context $\mathbf{u}$, let $G_\mathbf{u}(w)$ be the probability of the current word taking on value $w$. A PYP is used as the prior for $G_\mathbf{u}=[G_\mathbf{u}(w)]_{w\in W}$: 
  \begin{equation*}
  	G_\mathbf{u}\sim\operatorname{PY}(d_{|\mathbf{u}|}, \theta_{|\mathbf{u}|},G_{\pi(\mathbf{u})}),
  \end{equation*}
where $\pi(\mathbf{u})$ is the suffix of $\mathbf{u}$ consisting of all but the first word, and $|\mathbf{u}|$ being the length of $\mathbf{u}$. The priors are recursively placed with parameters $\theta_{|\pi(\mathbf{u})|}$, $d_{|\pi(\mathbf{u})|}$ and mean vector $G_{\pi(\pi(\mathbf{u}))}$, until we get to $G_\emptyset$: 		\begin{equation*}
		G_\emptyset \sim\operatorname{PY}(d_0,\theta_0,G_0),
    \end{equation*}\vspace{-0.05cm}
with $G_0$ being the uniformly distributed global mean vector for the empty context $\emptyset$.

\section{Backoff strategies}

Although numerous backoff strategies for $n$-grams have been proposed\cite{chen1996empirical}, due to their nature, skipgrams allow for even a greater number of backoff strategies. In this section we limit ourselves to three backoff strategies that we will investigate: \BON, \BOL, and \BOF. First we will introduce a graphical representation, followed by a formal definition.

The \BON backoff method uses the backoff structure as prescribed by the HPYPLM. Even when trained on skipgrams, \BON only backs off to $n$-grams. \Cref{fig:bon} illustrates the \BON hierarchy.

\begin{figure*}
	\begin{tikzpicture}[every node/.style={node distance=10pt, rectangle, minimum size=1cm}]
	\node[rectangle, minimum size=2cm](abcde) {abcde};
	\node[rectangle, minimum size=2cm, right=of abcde](bcde) {bcde};
	\node[rectangle, minimum size=2cm, right=of bcde](cde) {cde};
	\node[rectangle, minimum size=2cm, right=of cde](de) {de};
	\node[rectangle, minimum size=2cm, right=of de](e) {e};
	\draw[yshift=-2cm] (abcde) -> (bcde);
    \draw (bcde) -> (cde);
    \draw (cde) -> (de);
    \draw (de) -> (e);
	\end{tikzpicture}
    \caption{The backoff scheme for \BON. The example $5$-gram reads \emph{Bananas are my favourite fruit}, which we represent (for sake of brevity) here as abcde, with \emph{Bananas} as a, and \emph{fruit} as e, respectively. }\label{fig:bon}
    \end{figure*}


\BOF is similar to \BON, in that it always performs full recursive backoff to the a-priori unigram probabilities. But instead of ignoring the skipgrams, \BOF uses both pattern types. In contrast to the straight hierarchical backoff structure of \BON, \BOF yields a graph-like structure, starting with the root, $\mathbf{u}w$, and all steps ending in $\emptyset$. The structure is depicted in \cref{fig:bof}.

\begin{figure*}
    \begin{tikzpicture}[every node/.style={node distance=10pt, rectangle, minimum size=1cm}]
	\node[rectangle, minimum size=1cm](abcde) {abcde};
%     % I [above right=0.7cm and 4cm of A]
    \node[above left=0cm and 1.2cm of abcde](abcxe) {abc\{1\}e};
    \node[above=of abcxe](axcxe1) {a\{1\}c\{1\}e};
    \node[left=of axcxe1](bcxe1) {bc\{1\}e};
    \node[right=of axcxe1](abxxe1) {ab\{2\}e};
    \node[above=of bcxe1](cxe1) {c\{1\} e};
    \node[above=of axcxe1](bxxe1) {b\{2\}e};
    \node[above=of abxxe1](axxxe1) {a\{3\}e};
    \node[above=of bxxe1](e1) {e};
    \node[above=of e1](empty1) {.};
   %\draw (abxde) -> (bxde1);
    \draw (abcxe) -> (abxxe1);
    \draw (abcxe) -> (axcxe1);
    \draw (abcxe) -> (bcxe1);
    \draw (abxxe1) -> (axxxe1);
    \draw (axcxe1) -> (axxxe1);
    \draw (abxxe1) -> (bxxe1);
    \draw (bcxe1) -> (bxxe1);
    \draw (axcxe1) -> (cxe1);
    \draw (bcxe1) -> (cxe1);
    \draw (axxxe1) -> (e1);
    \draw (bxxe1) -> (e1);
    \draw (cxe1) -> (e1);
    \draw (e1) -> (empty1);
%     % II
    \node[above right=0cm and 0.8cm of abcde](axcde) {a\{1\}cde};
    \node[above=of axcde](axxde1) {a\{2\}de};
    \node[left=of axxde1](cde1) {cde};
    \node[right=of axxde1](axcxe2) {a\{1\}c\{1\}e};   
    \node[above=of axxde1](cxe2) {c\{1\}e};
    \node[left=of cxe2](de1) {de};
    \node[right=of cxe2](axxxe2) {a\{3\}e}; %
    \node[above=of cxe2](e2) {e};
    \node[above=of e2](empty2) {.};
    \draw (abxde) -> (bxde1);
    \draw (axcde) -> (axcxe2);
    \draw (axcde) -> (axxde1);
    \draw (axcde) -> (cde1);
    \draw (axcxe2) -> (axxxe2);
    \draw (axxde1) -> (axxxe2);
    \draw (axxde1) -> (de1);
    \draw (cde1) -> (de1);
    \draw (cde1) -> (cxe2);
    \draw (cxe2) -> (e2);
    \draw (de1) -> (e2);
    \draw (axxxe2) -> (e2);
    \draw (axcxe2) -> (cxe2);
    \draw (e2) -> (empty2);
    % III %% below
    \node[above left=0cm and 5.7cm of abcde](abxde) {ab\{1\}de};
    \node[above=of abxde](axxde2) {a\{2\}de};
    \node[left=of axxde2](bxde1) {b\{1\}de};
    \node[right=of axxde2](abxxe2) {ab\{2\}e};
    \node[above = of $(bxde1)!0.5!(axxde2)$](de2) {de};
    \node[above = of $(axxde2)!0.5!(abxxe2)$](bxxe2) {b\{2\}e};
    \node[above = of $(de2)!0.5!(bxxe2)$](e3) {e};
    \node[above = of e3](empty3) {.};
    \draw (abxde) -> (bxde1);
    \draw (abxde) -> (axxde2);
    \draw (abxde) -> (abxxe2);
    \draw (bxde1) -> (de2);
    \draw (axxde2) -> (de2);
    \draw (abxxe2) -> (bxxe2);
    \draw (de2) -> (e3);
    \draw (bxxe2) -> (e3);
    \draw (e3) -> (empty3);
    % IV %% below
    \node[above right=0cm and 6cm of abcde](bcde) {bcde};
    \node[above=of bcde](bcxe2) {bc\{1\}e};
    \node[left=of bcxe2](bxde2) {b\{1\}de};
    \node[right=of bcxe2](cde2) {cde};
    \node[above=of bcxe2](de3) {de};
    \node[left=of de3](bxxe3) {b\{2\}e};
    \node[right=of de3](cxe3) {c\{1\}e};
    \node[above=of de3](e4) {e};
    \node[above=of e4](empty4) {.};
    \draw (bcde) -> (bxde2);
    \draw (bcde) -> (bcxe2);
    \draw (bcde) -> (cde2);
    \draw (bxde2) -> (bxxe3);
    \draw (bcxe2) -> (bxxe3);
    \draw (bcxe2) -> (cxe3);
    \draw (bxde2) -> (de3);
    \draw (cde2) -> (de3);
    \draw (cde2) -> (cxe3);
    \draw (bxxe3) -> (e4);
    \draw (de3) -> (e4);
    \draw (cxe3) -> (e4);
    \draw (e4) -> (empty4);
    %
    \draw (abcde) -> (abcxe);
    \draw (abcde) -> (bcde);
    \draw (abcde) -> (axcde);
    \draw (abcde) -> (abxde);
	\end{tikzpicture}
    \caption{Following the same example as \cref{fig:bon}, we now illustrate all backoff paths for \BOF. Some steps such as de, c\{1\}e (inter alia), occur more than one time. However, depicting them only once would render the graph illegible. }\label{fig:bof}
    \end{figure*}

Each step we only allow one skip to be introduced, where the first and last\footnote{The focus word $w$.} cannot be a skip. \textbf{HIER een STUKJE over RATIOS}. The intuition is that if a pattern occurs in the data, we do not have to backoff, since we can already estimate a probability.

The \BOL backoff strategy is an extension of the \BOF backoff strategy, in that it also uses both $n$-grams and skipgrams and follows the same backoff rules, that stops the recursion if a test pattern $\mathbf{u}w$ has already occurred in the training data. This means that the count is not zero, and hence at training time a probability has been assigned to that pattern.

Suppose we have a collected the frequencies of patterns from \obw:\footnote{With the HPYPLM you would in fact look at the frequencies after sampling, since they might not be same as those in the original corpus. Especially in those cases where the sampled frequencies are now zero, or vice versa.}
\begin{multicols}{2}\begin{enumerate}
	\item[0] bananas are my favourite fruit
	\item[1] are my favourite fruit
	\item[2] my favourite fruit 
	\item[7] favourite fruit
	\item[15391] fruit
	\item[1] are \{1\} favourite fruit
	\item[1] are my \{1\} fruit
	\item[90] are \{2\} fruit
	\item[13] my \{1\} fruit
	\item[0] bananas \{1\} my \{1\} fruit
	\item[0] bananas \{1\} my favourite fruit
	\item[0] bananas \{2\} favourite fruit
	\item[4] bananas \{3\} fruit
	\item[0] bananas are \{1\} favourite fruit
	\item[0] bananas are \{2\} fruit
	\item[0] bananas are my \{1\} fruit
\end{enumerate}\label{enum:minicorpus}\end{multicols}


Then for all patterns in \cref{fig:bol} that occur in this corpus, we remove their backoff steps.

\begin{figure*}
    \begin{tikzpicture}[every node/.style={node distance=10pt, rectangle, minimum size=1cm}]
	\node[rectangle, minimum size=1cm](abcde) {abcde};
%     % I [above right=0.7cm and 4cm of A]
    \node[above left=0cm and 1cm of abcde](abcxe) {abc\{1\}e};
    \node[above=of abcxe](axcxe1) {a\{1\}c\{1\}e};
    \node[left=of axcxe1](bcxe1) {bc\{1\}e};
    \node[right=of axcxe1](abxxe1) {ab\{2\}e};
    \node[above=of bcxe1](cxe1) {c\{1\} e};
    \node[above=of axcxe1](bxxe1) {b\{2\}e};
    \node[above=of abxxe1](axxxe1) {a\{3\}e};
   %\draw (abxde) -> (bxde1);
    \draw (abcxe) -> (abxxe1);
    \draw (abcxe) -> (axcxe1);
    \draw (abcxe) -> (bcxe1);
    \draw (abxxe1) -> (axxxe1);
    \draw (axcxe1) -> (axxxe1);
    \draw (abxxe1) -> (bxxe1);
    \draw (axcxe1) -> (cxe1);
%     % II
    \node[above right=0cm and 1cm of abcde](axcde) {a\{1\}cde};
    \node[above=of axcde](axxde1) {a\{2\}de};
    \node[left=of axxde1](cde1) {cde};
    \node[right=of axxde1](axcxe2) {a\{1\}c\{1\}e};   
    \node[above=of axxde1](cxe2) {c\{1\}e};
    \node[left=of cxe2](de1) {de};
    \node[right=of cxe2](axxxe2) {a\{3\}e};
    \draw (abxde) -> (bxde1);
    \draw (axcde) -> (axcxe2);
    \draw (axcde) -> (axxde1);
    \draw (axcde) -> (cde1);
    \draw (axcxe2) -> (axxxe2);
    \draw (axxde1) -> (axxxe2);
    \draw (axxde1) -> (de1);
    \draw (axcxe2) -> (cxe2);
    % III %% below
    \node[above left=0cm and 5.5cm of abcde](abxde) {ab\{1\}de};
    \node[above=of abxde](axxde2) {a\{2\}de};
    \node[left=of axxde2](bxde1) {b\{1\}de};
    \node[right=of axxde2](abxxe2) {ab\{2\}e};
    \node[above = of $(bxde1)!0.5!(axxde2)$](de2) {de};
    \node[above = of $(axxde2)!0.5!(abxxe2)$](bxxe2) {b\{2\}e};
    \draw (abxde) -> (bxde1);
    \draw (abxde) -> (axxde2);
    \draw (abxde) -> (abxxe2);
    \draw (axxde2) -> (de2);
    \draw (abxxe2) -> (bxxe2);
    % IV %% below
    \node[above right=0cm and 5.5cm of abcde](bcde) {bcde};
    %
    \draw (abcde) -> (abcxe);
    \draw (abcde) -> (bcde);
    \draw (abcde) -> (axcde);
    \draw (abcde) -> (abxde);
	\end{tikzpicture}
    \caption{Following the counts on  \cpageref{enum:minicorpus} we pruned the backoff graph from \cref{fig:bof}, which results in the backoff graph for \BOL. For each step the backoff procedure stops when it has occurred in the training data.}\label{fig:bol}
    \end{figure*}

In a setup with only $n$-gram features, \BON and \BOF are functionally the same in terms of backoff strategy. The nuance lies in the way they do the backoff. \BOF tries to backoff to skipgrams,\footnote{It tries, does not find the skipgram, since it is not trained on skipgrams, and continues with the backoff procedure.} independent of whether it was trained with skipgrams. The \BON strategy has only one way to get to the a-priori word probability. 
For \BOF it can get to the word probabilities for each skipgram that is generated, hence it puts more emphasis on the word probabilities.\footnote{Compare the number of paths to e in \cref{fig:bon} and \cref{fig:bof}.}

Now for the formal definition of the strategies, for all strategies, we have that $p(w|\mathbf{u})=G_0(w)$ if $\mathbf{u} = \emptyset$. For \BON, the other case is defined as:
  \begin{equation}\begin{split}
  	p(w|\mathbf{u})= &
\frac{c_{\mathbf{u}w\cdot}-d_{|\mathbf{u}|}t_{\mathbf{u}w\cdot}}{\theta_{|\mathbf{u}|}+c_{\mathbf{u}\cdot\cdot}} +
\frac{\theta_{|\mathbf{u}|}+d_{|\mathbf{u}|}t_{\mathbf{u}\cdot\cdot}}{\theta_{|\mathbf{u}|}+c_{\mathbf{u}\cdot\cdot}}
p(w|\pi(\mathbf{u}))
  \end{split}\end{equation}
%with $c_{\mathbf{u}wk}$ being the count of $w$ having the value $k$ after context $\mathbf{u}$; 
with $c_{\mathbf{u}w\cdot}$ being the number of $\mathbf{u}w$ tokens, and $c_{\mathbf{u}\cdot\cdot}$ the number of patterns starting with context $\mathbf{u}$. Similarly, $t_{\mathbf{u}wk}$ is 1 if the $k$th draw from $G_{\mathbf{u}}$ was $w$, 0 otherwise. $t_{\mathbf{u}w\cdot}$ then denotes if there is a pattern $\mathbf{u}w$, and $t_{\mathbf{u}\cdot\cdot}$ is the number of types following context $\mathbf{u}$.
  
Recall\footnote{See \cref{ch:languagemodels}.} that $\sigma_n$ is the operator that adds a skip to a pattern $\mathbf{u}$ on the $n$th position if there is not already a skip. Then $\boldsymbol\sigma(\mathbf{u}) = \left[\sigma_n(\mathbf{u})\right]_{n=2}^{|\mathbf{u}|}$ is the set of patterns with one skip more than the number of skips currently in $\mathbf{u}$. The number of generated patterns is $\boldsymbol\varsigma=|\boldsymbol\sigma(\mathbf{u})|$.
We also introduce the indicator function $S$, which for the {\sf full} backoff strategy always returns its argument: $S_{\mathbf{u}w}(y) = y$.
%
%
The {\sf full} backoff strategy is defined as follows, with $\mathbf{u}_x = \boldsymbol\sigma_x(\mathbf{u})$, and discount frequency $\delta_{\mathbf{u}} = 1$:
 \begin{equation}
 \begin{split}
p(w|\mathbf{u}) =& \sum_{m=1}^{\boldsymbol\varsigma}\left\{ \frac{1}{\mathbf{\boldsymbol\varsigma}+1}\left[
\frac{c_{\mathbf{u}_mw\cdot}-\delta_{\mathbf{u}_m}d_{|\mathbf{u}_m|}t_{\mathbf{u}_mw\cdot}}{\delta_{\mathbf{u}_m}\theta_{|\mathbf{u}_m|}+c_{\mathbf{u}_m\cdot\cdot}} + S_{\mathbf{u}_mw}\left(
\frac{\theta_{|\mathbf{u}_m|}+d_{|\mathbf{u}_m|}t_{\mathbf{u}_m\cdot\cdot}}{\delta_{\mathbf{u}_m}\theta_{|\mathbf{u}_m|}+c_{\mathbf{u}_m\cdot\cdot}}
p(w|\pi(\mathbf{u}_m))\right)\right] \right\} \\ 
+ 
%%
& \frac{1}{\mathbf{\boldsymbol\varsigma}+1}\left[
\frac{c_{\mathbf{u}w\cdot}-\delta_{\mathbf{u}}d_{|\mathbf{u}|}t_{\mathbf{u}w\cdot}}{\delta_{\mathbf{u}}\theta_{|\mathbf{u}|}+c_{\mathbf{u}\cdot\cdot}}+ S_{\mathbf{u}w}\left(\frac{\theta_{|\mathbf{u}|}+d_{|\mathbf{u}|}t_{\mathbf{u}\cdot\cdot}}{\delta_{\mathbf{u}}\theta_{|\mathbf{u}|}+c_{\mathbf{u}\cdot\cdot}}
p(w|\pi(\mathbf{u}))\right)\right]
  \end{split}\end{equation}
 
 The \BOL backoff strategy is an extension of the \BOF backoff strategy that stops the recursion if a test pattern $\mathbf{u}w$ has already occurred in the training data. This means that the count is not zero, and hence at training time a probability has been assigned to that pattern. $S$ is the indicator function which tells if a pattern has been seen during training: $S_{\mathbf{u}w}(\cdot) = 0$ if $\mathrm{count}(\mathbf{u}w) > 0$, $1$ otherwise; and $\delta_{\mathbf{u}} = V-\sum_{w\in W} S_{\mathbf{u}w}(\cdot)$. Setting $S_{\mathbf{u}w}(\cdot) = 0$ stops the recursion.

The backoff strategies \BON and \BOF always use the same number of backoff probabilities, for $5$-grams 5 and 53 respectively\footnote{For $4$-grams, 5 and 22 respectively.}. For \BOL this is dependent on the training material. If a top level pattern matches the training material, then only one probability is used. In the worst scenario, \BOL equals \BOF.

\section{Experimental Setup}

We train $4$-gram language models with the HPYPLM as described in this chapter, on three corpora as described in \cref{chap:data}. In this chapter we do not use sentences markers to denote the begin or end of a sentence. Since we do not compute the perplexity on sentence-level, but rather on individual patterns that comprise the test corpus.

At the core of our experimental framework we use \texttt{cpyp}\footnote{\url{https://github.com/redpony/cpyp}}, which is an existing library for non-parametric Bayesian modelling with Pitman-Yor priors with histogram-based sampling\cite{blunsom2009note}. This library has an example application to showcase its performance with $n$-gram-based language modelling. Limitations of the library, such as not natively supporting skip- and flexgrams, and other functionality, such as thresholding and discarding of certain patterns, led us to extend the library with \texttt{Colibri Core}\footnote{\url{https://proycon.github.io/colibri-core/}}, a pattern modelling library. \texttt{Colibri Core} can handle these limitations\cite{gompel2016efficient}, and together the libraries are a complete Bayesian language model that handles skipgrams.\footnote{The code and documentation is available at \url{https://github.com/naiaden/SLM}.}.

Each model is run for 50 iterations\footnote{Without an explicit burn-in phase, as we are not using statistics or samples from any but the last iteration.}, with hyperparameters $\theta = 1.0$, and $\gamma = 0.8$. The hyperparameters are resampled every 30 iterations\footnote{Which in practice results in one resampling of the hyperparameters.} with slice sampling\footnote{walker2007sampling}.

We test each model on different test sets, and we collect their intrinsic performance by means of perplexity. For the words in the test set that were unseen in the training data, we discard their probability.

\section{Results}
In this section we describe the results of multiple experiments. First we compare the HPYPLM approach to a traditional $n$-gram language model. We then show the influence of using skipgrams in addition to $n$-grams, and investigate whether this influence holds in a cross-domain evaluation. Second, we compare the backoff strategies, and investigate which domains fit which backoff strategy best.
\clearpage
\cite{pickhardt2014generalized} report perplexity values on two subsets of the test material of jrc-p and wp-p\footnote{\textbf{Tell something about this!!}}. Of particular interest are the values for $n =4$, where for training on \wp and testing on wp-p a perplexity of 404 is reported with MKN smoothing, and 378 with their generalised language model (GLM). Similarly for the \jrc and jrc-p, they show a reduction of perplexity from 83 with MKN to 81 with their GLM. In our attempts to reproduce these values with SRILM\cite{stolcke2002srilm}, we found substantially lower perplexity values for MKN,
\footnote{Lower perplexity values indicate a higher intrinsic performance, which is better.} as can be seen in \cref{tab:mknvsglm}.
\footnote{We used the following program call: \texttt{ngram-count -kndiscount -gt3min 1 -gt4min 1 -order 4 -interpolate -kndiscount -unk}.}
 The rows represent the train corpora; columns represent the test corpora. Comparing our results with the perplexity values reported for the GLM thus yields an unfair comparison. For this reason we choose the baseline to be the perplexity values from SRILM.

\begin{table*}
	\begin{tabular}{l*{6}{l}l*{6}{l}}
		&  \multicolumn{6}{c}{HPYLM} & &  \multicolumn{6}{c}{modified Kneser-Ney} \\
		& \jrc	& \obw	& \emea	&	\wp	& jrc-p	& wp-p & & \jrc	& \obw &	\emea & \wp&  jrc-p & wp-p\\ \cline{2-7}\cline{9-14}
		\jrc	& 13 & 1510 & 1081 & 1293 & 13 & 1269 & & 22 & 1664 & 1310 & 1837 & 122 & 1736 \\
		\obw & 1232 & 171 & 1749 & 724 & 1156 & 692 & & 1460 & 211 & 1516 & 996 & 1383 & 1046 \\
		\emea & 769 & 1552 & 4 & 1097 & 774 & 1093 & & 1115 & 1745 & 10 &1669 & 993 & 1444 \\
		wps & 555 & 455 & 1005 & 217 & 537 & 212 & & 842 & 598 & 1590 & 449 & 1088 & 646
	\end{tabular}
	\caption{A side-by-side comparison of perplexities of two $n$-gram language models with $n=4$: the Bayesian hierarchical Pitman-Yor language model (HPYLM) and the frequentist modified Kneser-Ney, as implemented by SRILM. The rows represent the training corpora, and the columns the test sets. }\label{tab:mknvsglm}
		%The two additional data sets jrc-p and wp-p are around 100k randomly selected 4-grams from jrc and wp respectively, as selected by \newcite{pickhardt2014generalized}. 
\end{table*}

\cite{teh2006hierarchical} shows that the HPYPLM can outperform MKN. In our experiments we reach the same conclusion, on the basis of the results listed in \cref{tab:mknvsglm}. On the right we show the results with SRILM's implementation of MKN, now without sentence boundaries, equal to the settings of our HPYPLM. The perplexities obtained by the HPYPLM are consistently lower by a substantial margin.

Bayesian language models are more complex than their frequentist counterparts such as MKN. Not only the model itself is more complex, but also the computational complexity, both in terms of time and memory. On an Intel Xeon E5-2660 with 512G working memory it takes 3 days to compute the \obw model, and serialize the model to file. However, if we only use one iteration for sampling,\footnote{With zero sampling iterations, we have effectively reduced the HPYPLM to IKN.} we already achieve perplexities approaching the perplexities reported in \cref{tab:mknvsglm} with 50 iterations, in all setups. This one-iteration learning process does not alleviate the burden of memory complexity\footnote{The number of parameters only changes when a restaurant has no visitors, or when one is being founded. The number of customers stays constant over all sampling iterations.}, but training is now only a factor two slower than traditional MKN, and the performance is already notably better.
Yet, in the remainder of this chapter we use 50 iterations for Gibbs sampling. For testing, once the model has been loaded into memory, there is no time penalty involved when choosing HPYPLM over MKN, which is an important property for extrinsic and time-dependent evaluation.

\subsection{Influence of adding skipgrams to an $n$-gram language model}
The results for English language models with skipgrams are reported in \cref{ta:ngramvsskipgram} in terms of perplexity. The rows represent the train corpora, and columns the test corpora. On the diagonals the perplexity always has the lowest value for that column, since the best performance on a test set is achieved by training on the same domain, for all domains. 

We also observe an effect of domain-specific versus generic corpora. The within-domain performance of domain-specific corpora yields very low perplexities (13 on \jrc, 4 on \emea), while these are higher for within-domain evaluation for generic corpora (158 for \obw, 216 for \wp). There is no difference between training with only $n$-grams and training with skipgrams on within-domain test sets.

     \begin{table*}
	\begin{tabular}{lllllllllllllll}
		& \multicolumn{4}{c}{$n$-grams}	& & \multicolumn{4}{c}{skipgrams}&& \multicolumn{4}{c}{Relative difference (in \%)}\\
		& \jrc	& \obw	& \emea	& \wp	& & \jrc	& \obw	& \emea	& \wp	& & \jrc	& \obw	& \emea	& \wp \\ \cline{2-5}\cline{7-10} \cline{12-15}
		\jrc		& 13	& 1195	& 961	& 1011	& & 13	& 1162	& 939	& 1008	& & 0	& -3	& -2	& 0		\\
		\obw	& 1232	& 171	& 1749	& 724	& & 728	& 141	& 1069	& 542	& & -41	& -6	& -39	& -25	\\
		\emea	& 600	& 1143	& 4		& 843	& & 581	& 1155	& 4		& 842	& & -3	& 1	& 1		& 0		\\
		wps		& 555	& 455	& 1005	& 217	& & 565	& 470	& 990	& 227	& & 2	& 3	& -1	& 4
	\end{tabular} 
	\caption{An overview of the results to compare the influence and contribution of skipgrams for English. 
		The perplexities can only be compared row-wise, as the vocabulary depends on the training set. For these values we chose the best-performing backoff strategy. The relative difference is the percentual change, with a negative value indicating an improvement with skipgrams.}\label{ta:ngramvsskipgram}
\end{table*} 

However, we find that adding skipgrams does tend to contribute to a lower perplexity in a cross-domain setting, with many non-diagonal cells showing lower perplexities. We observe absolute perplexity reductions up to 33, and relative reductions up to 3.2\%. We are mostly interested in the effects of training on a generic corpus, and testing on a domain-specific corpus: it is generally easier to obtain generic texts, and even if there is a corpus for that specific domain, there will always be more generic data. If it turns out that the generic corpora can be used to model domain-specific corpora without much deterioration, this paves the way to fully use the billion-word data sets that are currently available for many languages.\footnote{COW CORPUS} If we focus on the rows for \obw and \wp, we see that for the cross-domain setups with \jrc and \emea, the performance improves when we add skipgrams to the input features, except when we test \jrc trained on \wp.

In particular we focus on \obw for the remainder of this thesis, since compared to \wp, which contains articles in Wikipedia's encyclopaedic style, it contains a wider variety of text, and is less restricted in language use. Supported by the perplexity results, we consider \wp to be less generic.\footnote{This is independent of the number of words.} This is reflected in the perplexity scores as well, where a within-domain experiment on \wp (227 ppl) appears to be more difficult than on \obw (141 ppl). In the \obw-\wp and \wp-\obw experiments the perplexities are more in range, 542 and 470 respectively.

\subsection{Effect of different backoff methods}
In the previous subsection we compared pure $n$-gram models with skipgram models, where we used the best performing backoff methods. In this subsection we study the effect of different backoff methods in more detail. 

Although skipgram models also contain all $n$-grams as an $n$-gram model trained on the same data, the $n$-gram probabilities differ between the two models. If we train with skipgram features, but only use the $n$-gram during testing (\BON), one would expect the same perplexity. However, a side-effect of training with skipgrams is that the lower-order patterns contained in the skipgrams are seen more often, compared to training with only $n$-grams.

In our analyses we included this strategy to see the negative effects of using skipgrams, because even in scenarios where they do not contribute, \BON may still yield a perplexity similar to HPYPLM trained on only $n$-grams. If that is the case, we can opt to use skipgrams whenever possible, without risking a deterioration in performance.

The \BOF backoff strategy is effective especially in cross-domain situations, where backing off to the word probabilities is beneficial. We see this especially in \jrc and \emea; for \wp this effect is not present. If we compare the OOV rate for \wp on cross-domain sets, it is consistently higher than the OOV rate for \obw on other sets. This explains why a backoff procedure which emphasises the word level more does not help in such cases.

For a within-domain experiment on \obw, we notice that throughout the learning curve until 30M words, \BOF is slightly outperforming \BOL. But with more training data, \BOL has seen enough $n$-grams to overtake \BOF.


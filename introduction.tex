\chapter{Introduction to the thesis}

\section{Motivation}

\section{Scientific relevance}

\section{Societal relevance}






\section{Research questions}
The main question that drives the studies in this thesis is:
To what extent can skipgrams, as generalisations of $n$-grams, contribute to a better performance in $n$-gram-based language models?

helpen skipgrams:
\begin{itemize}
	\item intrinsiek
	\item extrinsiek
\end{itemize}

\section{Research methodology}

\section{Thesis contributions}
This thesis contains but 1 published paper, aims to investigate an old idea, with new computational possibilities.

\section{Thesis outline}
The remainder of this thesis consists of three introductory chapters, introducing frequentist $n$-gram and skipgram language modelling \cref{chap:introlm}, Bayesian $n$-gram and skipgram language modelling \cref{chap:introblm}, and finally an introduction to the data used in this thesis.\footnote{Er is iets fishy met de referenties hier}

What follows are two parts, the first describing an intrinsic evaluation of the skipgram language models (\cref{chap:shpyplm}), whereas in the second part (\cref{ch:speech}) we consider extrinsic evaluation on automatic speech recognition.

In the final chapter we present our conclusions, and stipulate ideas for future work.

